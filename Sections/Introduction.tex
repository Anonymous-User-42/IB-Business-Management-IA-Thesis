{DDE is a private construction contracting company with one shareholder based in Dubai and operates on construction projects throughout the United Arab Emirates on a contract basis. It is involved in supplying and procuring labour, material, equipment, and general miscellaneous technical works related to the construction of general structures and buildings.}

{The company currently rents an office in Al Nahda, a district in Dubai, with a total of 3 employees. The primary customers of this firm are usually consultants or other bigger contractors (those who handle more extensive project/construction works). The company DDE receives payment of their work (per month) three months after they file for a payment (three months after work is done). This method of payment is generally accepted as an industry standard.}

{This negatively impacts the company when the company is to pay their monthly and quarterly expenses as the profit the company has in their books is still not in hand. This method of payment generally does not create cash flow issues. However, if the company is to invest their capital in procuring/advancing certain business segments, this method of advance payment creates a problem with respect to cash flow.}

{DDE intends to expand their market share and increase its revenue generated by outsourcing its labour demands to other small sub-contracting companies. When outsourcing labour in this regard, DDE needs to issue payments on a monthly basis without the 3-month delay. To effectively employ this new strategy, the company needs to have enough money in hand to issue payments to these outsourced companies before payments from the main companies are received.}

{If their plan is successful, it will help DDE have an extra, alternative source of revenue and enable them to undertake more construction projects as they would not need to use the company labour, rather, outsourced labour.}

{From this arises the research question, \textbf{"Should the company, DDE, consider outsourcing their labour demands ?"}}

